\section{Simulation Studies}
\label{sec:mc}

In this section, we introduce several shower quality variables to estimate
the quality of shower reconstruction in~\Cref{sec:shr_quality}, and examine
shower reconstruction with simulated single particle Monte Carlo samples
in~\Cref{sec:single_em,sec:single_pi0}.

% -----------------------------------------------------------------------
\subsection{Shower Quality}
\label{sec:shr_quality}

We make plots of variables of reconstructed shower profile and compare
some of them with simulation.
The object MC shower, which describes the truth level of detectable shower
profile, or the best reconstruction quality we could achieve, 
is used as the reference point of the comparison.
The details about MC shower can be found in~\cite{DocDB3771}.
Specifically, the direction of MC showers typically represents the truth 
direction of the original electromagnetic particle, while the energy of
MC shower accounts for that deposited in the detector.\\
\\
The shower quality variables are listed below,
\begin{itemize}
\item 3D angle difference, the angular difference in the shower direction 
      between
      MC and reconstructed showers in 3D.
\item Shower length, the length of reconstructed showers, as described
      in~\Cref{sec:shr_geometry}.
\item Starting point difference, the 3D distance between the shower
      starting point of MC shower and that of the reconstructed shower.
\item Clustering efficiency, the ratio of sum of the charges
      in a reconstructed shower to the total deposited charges in
      each wire plane.
\item Energy resolution, $\frac{\textrm{MC deposited energy}-\textrm{Reconstructed energy}}{\textrm{MC deposited energy}}$ in each wire plane.
\item Energy asymmetry, $\frac{\textrm{MC deposited energy}-\textrm{Reconstructed energy}}{\textrm{MC deposited energy}+\textrm{Reconstructed energy}}$ in each wire plane.
\item Energy correlation, a 2D histogram of reconstructed energy versus 
      the MC deposited energy in each wire plane.
\item dQ/dx, dE/dx, the quantities from shower reconstruction.
\item Number of MC/reconstructed showers, the number of MC/reconstructed
      showers in each event.
\end{itemize}

% -----------------------------------------------------------------------
\subsection{Single Electromagnetic Particle Sample}
\label{sec:single_em}

We start estimating shower reconstruction qualities with 10,000 simulated
single electron events.
To disentangle different sources of inefficiency, we also prepare a sample
without the bad wires; 
i.e. all the wires in the simulation are good.
The momentum and angle distributions are generated according to
the Laudau distribution,
\begin{equation}
\label{eq:bnb-like}
P(x) = \frac{1}{2\pi i}\int^{c+i\infty}_{c-i\infty}e^{xs+s\log s}ds,
\end{equation}
where, for the momentum distribution, $x = (p-0.3)/0.1$, and $p$ 
represents the modular momentum within a range of 0.1 to 1~GeV.
Similarly, the angular distribution is profiled by setting 
$x = (\theta-0.5)/0.2$,
where $\theta$ is the angle with respect to the $z$-axis ranging from
0 to $2\pi$. \\
\\
We use \texttt{larsoft} and \texttt{uboonecode} \texttt{v05\_08\_00} to
generate the simulated MC samples.
The reconstruction chain is identical to the configuration used in 
the official Monte Carlo challenge~7 (MCC~7), while the simulation 
might have included minor updates.
Nonetheless, there is no significant changes in simulation
since MCC~7.
The MC samples are subsequently larlited with \texttt{larlite} 
\texttt{v05\_02\_00}, and the shower reconstruction is performed in the
\texttt{larlite} framework.\\
\\
% Shower reconstruction
Only the shower-like PFParticles (with a PDG code 11) are selected
as shower candidates and reconstructed in this section.
We extend the selection, and the study is presented 
in~\Cref{sec:ongoing}. \\
\\
% Results
Fig.? shows good resolution on shower directions and shower starting
points.
In particular, the two variables are better reconstructed in
electron events, possibly owing to the fact that photons are
not detected before their pair production.\\
\\
Energy deficiency is manifested in Fig.?.
Possible sources may come from
\begin{itemize}
\item charge deposits below the threshold of the hit finder
\item missing charge deposits owing to bad wires
\item clustering inefficiency
\item energy calibration
\end{itemize}
We further discuss the sources in~\Cref{sec:ongoing}. 
The peaks at zero in all the clustering efficiency and calorimetry
plots illustrate the feature of Pandora that it is not necessary to
have information in all the three wire planes to reconstruct a 
PFParticle.


% -----------------------------------------------------------------------
\subsection{Single $\pi^0$ Sample}
\label{sec:single_pi0}

Reconstruction of the $\pi^0$ invariant mass is a benchmark of
the quality of shower reconstruction, as both the energy and
direction are involved,
\begin{equation}
\label{eq:pi0_mass}
m_{\pi^0} = \sqrt{2E_1 E_2(1-\cos\theta_{12})},
\end{equation}
where $E_1$ and $E_2$ represent the energy of the two photons
from the $\pi^0$ decay, and $\theta_{12}$ the angle between the
two photons.\\
\\
To examine the reconstruction of $\pi^0$ mass, a single $\pizero$
Monte Carlo sample is generated with the momentum and
angular distributions identical to those in the single electron
and photon samples detailed in~\Cref{sec:single_em}.
Moreover, the simulation and reconstruction configuration is
exactly the same; for instance,
the shower reconstruction is performed on shower-like PFParticles
in the plots illustrated here.\\
\\
The reconstructed showers are matched to the MC showers according
to the timing and the waveforms on the wires.
We select the events with the following criteria,
\begin{itemize}
\item containing exactly two reconstructed and matched showers,
\item both reconstructed energy greater than 0.1~MeV,
\item recontructed $\cos\theta_{12}$ less than 1,
\end{itemize}
and the $\pizero$ mass obtained from those two showers
are shown in~Fig.?.
In other words, the $\pizero$ mass is reconstructed from the perfect
combination of reconstructed showers.
A resonance peak is presented in the $\pizero$ mass distribution.\\
\\
% DCA, MC/reco combination to check the features of the pi0 mass 
% distributions
We further investigate the distance of closest approach between
the $\pizero$ decay point from the MC truth information and
the reconstructed shower direction.
Fig.? demonstrates the reconstructed shower direction is reasonable.
In addition, as shown in Fig.?,
we check the impact of the energy and direction by
combining quatities from MC showers
and reconstructed showers,
\begin{itemize}
\item MC truth energy and direction: a $\delta$-function at 134~MeV,
\item MC deposited energy and MC truth direction: the long tail
      represents the energy outside the detector,
\item MC deposited energy and reconstructed direction: the smeared
      direction results in the long tail in the $\pizero$ mass distribution,
\item reconstructed energy and MC truth direction: the deficit in
      the reconstructed energy results in the position of the peak,
      as well as the width of the distribution.
\end{itemize}

% -----------------------------------------------------------------------
\subsection{Full BNB Neutrinos and Cosmic Sample}
\label{sec:bnb}

We have not started studies on the simulated BNB neutrino samples.


