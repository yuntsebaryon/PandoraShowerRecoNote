\section{$\pi^{0}$ Mass Reconstruction in Data}
\label{sec:data_pi0}

% Samples
We use two data samples in this study,
\begin{itemize}
\item 12 events, \texttt{PiZeroROI} + handscanning
\item 6 events, NuMuCC filter + \texttt{PiZeroROI} + handscanning
\end{itemize}

All the shower-like PFParticles in the events in the data samples 
are reconstructed as described in~\Cref{sec:reco}.
The subsequent step would be to find the two showers decaying from a 
$\pizero$ so that the $\pizero$ mass can be reconstructed according
to~\Cref{eq:pi0_mass}. \\
\\
% -----------------------------------------------------------------------
% Find the two showers: PFParticle hierarchy
Owing to the selection criteria of $\pizeroroi$, there is exactly one
neutrino-like PFParticle with two or three shower-like daughters in
each event.
Utilizing the hierarchy feature of Pandora outputs thereby becomes
the most straightforward strategy.
We loop over all the neutrino-like PFParticles (PDG code 12 or 14), 
identifying the one with at least two shower-like daughters.
In most of the events, the neutrino-like PFParticle has exactly two 
shower-like daughters, and the $\pizero$ mass is directly evaluated
from those two showers.
In the case that there are three shower-like daughters, we loop over
all the combinations of the showers and select the one with the largest
reconstructed $\pizero$ mass.
This approach gives us a reasonable result in our high-purity samples
as shown in Fig.x.\\
\\
% -----------------------------------------------------------------------
% Ongoing issues
The most relevant topic on the current $\pizero$ mass reconstruction
in high-purity $\pizero$ samples is the deficiency in the reconstructed
energy, which shifts the peak of the $\pizero$ mass distribution.
The same feature is obviously seen in the studies with the single electron
and photon MC samples, as mentioned in~\Cref{sec:single_em}.
We discuss the ongoing studies and possible corrections in~\Cref{sec:ongoing}.


