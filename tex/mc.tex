\section{Simulation Studies}
\label{sec:mc}

In this section, we introduce several shower quality variables to estimate
the quality of shower reconstruction in~\Cref{sec:shr_quality}, and examine
shower reconstruction with simulated single particle Monte Carlo samples
in~\Cref{sec:single_em,sec:single_pi0}.

% -----------------------------------------------------------------------
\subsection{Shower Quality}
\label{sec:shr_quality}

We make plots of variables of reconstructed shower profile and compare
some of them with simulation.
The object MC shower, which describes the truth level of detectable shower
profile, or the best reconstruction quality we could achieve, 
is used as the reference point of the comparison.
The details about MC shower can be found in~\cite{DocDB3771}.
Specifically, the direction of MC showers typically represents the truth 
direction of the original electromagnetic particle, while the energy of
MC shower accounts for that deposited in the detector.\\
\\
The shower quality variables are listed below,
\begin{itemize}
\item 3D angle difference, the angular difference in the shower direction 
      between
      MC and reconstructed showers in 3D.
\item Shower length, the length of reconstructed showers, as described
      in~\Cref{sec:shr_geometry}.
\item Starting point difference, the 3D distance between the shower
      starting point of MC shower and that of the reconstructed shower.
\item Clustering efficiency, the ratio of sum of the charges
      in a reconstructed shower to the total deposited charges in
      each wire plane.
\item Energy resolution, $\frac{\textrm{MC deposited energy}-\textrm{Reconstructed energy}}{\textrm{MC deposited energy}}$ in each wire plane.
\item Energy asymmetry, $\frac{\textrm{MC deposited energy}-\textrm{Reconstructed energy}}{\textrm{MC deposited energy}+\textrm{Reconstructed energy}}$ in each wire plane.
\item Energy correlation, a 2D histogram of reconstructed energy versus 
      the MC deposited energy in each wire plane.
\item dQ/dx, dE/dx, the quantities from shower reconstruction.
\item Number of MC/reconstructed showers, the number of MC/reconstructed
      showers in each event.
\end{itemize}

% -----------------------------------------------------------------------
\subsection{Single Electromagnetic Particle Sample}
\label{sec:single_em}

We start estimating shower reconstruction qualities with 10,000 simulated
single electron events.
To disentangle different sources of inefficiency, we turn off the bad wire
configuration in this sample; i.e. all the wires in the simulation are good.
The ``BNB-like'' momentum and angle distributions are used to generate 
such events, where the Laudau distribution is used to describe the ``BNB-like''
features,
\begin{equation}
\label{eq:bnb-like-p}
P(x) = \frac{1}{2\pi i}\int^{c+i\infty}_{c-i\infty}e^{xs+s\log s}ds,
\end{equation}
where $x = (p-0.3)/0.1$, and $p$ represents the modular momentum within
a range of 0.1 to 1~GeV.


% -----------------------------------------------------------------------
\subsection{Single $\pi^0$ Sample}
\label{sec:single_pi0}

% -----------------------------------------------------------------------
\subsection{Full BNB Neutrinos and Cosmic Sample}
\label{sec:bnb}

We have not started studies on the simulated BNB neutrino samples.


