\section{Shower Reconstruction}
\label{sec:reco}

In this section we describe the underlying concepts for shower reconstruction.
The technical details are dispicted in DocDB ?.

% -----------------------------------------------------------------------
\subsection{Brief Introduction to Pandora}
\label{sec:pandora}

Pandora is a toolkit for pattern recognition, aiming to 
reconstruct particle flow in detectors with fine granularity, such as
detectors of high-energy lepton colliders, and LAr TPC.
MicroBooNE uses Pandora, which is integrated into the LArSoft toolkit
via the \texttt{larpandora} interface,
as one of the pattern recognition and clustering algorithms. \\
\\
% Concept of PFParticle
The outputs of Pandora are centralized in particle-flow particles, 
or PFParticles, which include,
\begin{itemize}
\item PDG code: mainly four types, 11 (shower-like),
      13 (track-like), 12 (neutrino-like PFParticle with a shower-like
      daughter), and 14 (neutrino-like PFParticle with a track-like daughter).
\item Parent: indicates the parent of the PFParticle.
\item Daughters: indicates the daughters (possibly a plural number) of
      the PFParticle.
\end{itemize}
Further, the PFParticles associate to the following objects created
by Pandora,
\begin{itemize}
\item Cluster: collection of hits from a separate hit finding algorithm.
\item Spacepoint: three-dimensional points indicating the 3D positions
      of charge deposition; associating to the two-dimensional hits
      reconstructed by the hit finder.
\item Vertex: a 3D point representing the starting point of the PFParticle.
\item Seed: not generated in a shower-like PFParticle
\item Track
\end{itemize}

In the following sections, we describe the algorithms of shower reconstruction 
utilizing these features.

% -----------------------------------------------------------------------
\subsection{Direction}
\label{sec:shrdir}

An important variable from shower reconstruction is the direction of the 
original electromagnetic particle.
We assume the direction of the vectorial summation of all showering particles 
would retain that of the original particle.
To obtain the direction,
we simply perform a 3-D principal component 
analysis (PCA) over all the spacepoint associated to a PFParticle.
The eigenvector corresponding to the largest eigenvalue determined by
PCA is taken to be the direction of the shower, as it represents the axis
with the largest possible variance of the distribution of the spacepoints.
The forward or backward direction is determined from the ordering of
the spacepoints, which is reconstructed based on the particle flow approach in
the Pandora toolkit.
Further, it will be revisited in the succeeding
shower reconstruction algorithms.\\
\\
In addtion, the (unweighted) 3D centroid of a PFParticle is evaluated
by PCA, and is saved for possible exploitation in other algorithms.

% -----------------------------------------------------------------------
\subsection{Geometry}
\label{sec:shrgeo}

% Length, width
The length and the 2D width of a shower is also determined by the same
principal component analysis.
As the largest eigenvalue of the PCA accounts for the possible largest
variance of the spacepoint distribution, the distance of three
standard deviations can be used to represent the length of a shower.
Moreover, the eigenvalues, or the variances, from PCA determine semi-axes
of the spacepoint distribution, and therefore a factor of two has to be
applied to obtain the full shower length.
\begin{equation}
\label{eq:shrlength}
\textrm{Shower Length} = 2\times 3\times \sqrt{\textrm{the largest eigenvalue from PCA}}
\end{equation}
Similarly, we use~\Cref{eq:shrlength} with the other two eigenvalues
from PCA to dispict the width of a shower.
Currently the widths in the plane orthogonal to the principal component
axis are not combined but stored individually.

% Opening angle
The opening angle of a shower can also simply approximated using the
eigenvalues of PCA.
A straightforward approach is to interpret the tangent of half the 
opening angle as the ratio of the standard deviations of the second
largest eigenvalue to the largest one, i.e.
\begin{equation}
\label{eq:shropeningangle}
\textrm{Opening Angle} = 2\times \tan^{-1}({\frac{\textrm{the 2nd largest eigenvalue}}{\textrm{the largest eigenvalue}}})
\end{equation}

% -----------------------------------------------------------------------
\subsection{Starting Point}
\label{sec:shrstartingpt}

% -----------------------------------------------------------------------
\subsection{Calorimetry}

