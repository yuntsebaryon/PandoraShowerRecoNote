\section{Ongoing Studies and Plans}
\label{sec:ongoing}

In this section, we discuss the ongoing studies on the deficit in
the reconstructed energy, and the proposed treatments which is being
tested.
In addition, we outline the plans on the shower reconstruction in
near future, which aim to cope with events with more complicated
topologies.

% -----------------------------------------------------------------------
\subsection{Energy Deficit}
\label{sec:energy_deficit}

As mentioned in~\Cref{sec:data_pi0}, the energy deficit in the reconstructed
showers has the greatest impact in the reconstructed $\pizero$ mass.
First of all, we try to disentangle the impact of clustering inefficiency
and that of other possible sources.
Starting with the single electron sample,
we merge all reconstructed hits in an event into a PFParticle
and reconstruct the energy of the PFParticle with the same algorithm
mentioned in~\Cref{sec:shr_calorimetry}.
Such a PFParticle represents the perfect clustering, as all the hits
in a single particle event come from an original particle.\\
\\
Fig.x displays the energy resolution (defined in~\Cref{sec:shr_quality})
in the three wire planes with the four combinations of the wire configuratios 
and the hit producers,
\begin{itemize}
\item Wire configuration
  \begin{itemize}
  \item all good wires (G): all the wires are good wires,
  \item bad wires (B): using the bad wire configuration according to
        the database,
  \end{itemize}
\item Hit producer
  \begin{itemize}
  \item gaushit (Gaushit): hits produced by the Gaussian hit finder;
        i.e. all the hits in an event,
  \item pandoraCosmicsKHitRemoval (PandoraCRm): hits produced by
        pandoraCosmics, which removes the hits from cosmic rays~\cite{DocDB5828}.
  \end{itemize}
\end{itemize}
The peak of energy resolution in all the combinations is offset
from zero by about 15\%, demonstrating the amount of the energy deficit
beyond clustering inefficiency.
This might attribute to energy calibration, and/or small amount of 
charge deposition which does not pass the threshold in the hit finder.
Furthermore, the existence of bad wires degrades the resolution, but
does not have a significant impact on the energy deficiency.

% -----------------------------------------------------------------------
\subsection{Clustering Inefficiency}
\label{sec:clustering_ineff}

Fig.x demonstrates (5-10)\% of energy deficit causing by clustering
inefficiency in addition to the 15\% described in~\Cref{sec:energy_deficit}.
It is noticable Pandora creates multiple small PFParticles in 
a single electron/photon event, and leaves a number of hits unclustered.
To retrieve the energy, we start with merging PFParticles, and then
perform a reclustering algorithm to collect the unclustered hits.
\Cref{sec:merging,sec:adding_hits} sketch the two steps respectively.

% -----------------------------------------------------------------------
\subsubsection{Merging PFParticles}
\label{sec:merging}

Not only does Pandora create multiple small PFParticles in
the single electron, photon samples,
but it also identifies a number of PFParticles as tracks.
Aiming to investigate how much energy would be retrieved by merging
multiple PFParticles, we include both shower-like and track-like
PFParticles in this study.\\
\\
The current merging algorithm is designed to cope with ... scattering
observed by handscanning.

\begin{itemize}
\item the vertex of the PFParticle falls within its cone
\item the principal component axis of the PFParticle is aligned
      within the cone
\end{itemize}

% -----------------------------------------------------------------------
\subsubsection{Adding Unclustered Hits}
\label{sec:adding_hits}

% -----------------------------------------------------------------------
\subsection{Track-Shower Separation}

% -----------------------------------------------------------------------
\subsection{Electron-Photon Separation}
