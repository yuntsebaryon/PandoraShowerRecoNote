\section{Shower Reconstruction}

In this section we describe the underlying concepts for shower reconstruction.
The technical details are dispicted in DocDB ?.

% -----------------------------------------------------------------------
\subsection{Brief Introduction to Pandora}

Pandora is a toolkit for pattern recognition, aiming to 
reconstruct particle flow in detectors with fine granularity, such as
detectors of high-energy lepton colliders, and LAr TPC.
MicroBooNE uses Pandora, which is integrated into the LArSoft toolkit
via the \texttt{larpandora} interface,
as one of the pattern recognition and clustering algorithms. \\
\\
% Concept of PFParticle
The outputs of Pandora are centralized in particle-flow particles, 
or PFParticles, which include,
\begin{itemize}
\item PDG code
\item Parent
\item Daughters
\end{itemize}
Further, the PFParticles associate to the following objects created
by Pandora,
\begin{itemize}
\item Cluster
\item Spacepoint
\item Vertex
\item Seed
\item Track
\end{itemize}

In the following sections, the algorithms of shower reconstruction 
utilizing these features are described.

% -----------------------------------------------------------------------
\subsection{Direction}

With the spacepoints, the three-dimensional points of charge deposition
reconstructed by Pandora, we simply perform a 3-D principal component 
analysis (PCA).
The eigenvector corresponding to the largest eigenvalue determined from
PCA is taken as the direction of the shower, as it represents the axis
with the largest possible variance of the distribution of the spacepoints.
The forward or backward direction is determined from the ordering of
the spacepoints, which is included in the particle flow approach in
the Pandora toolkit.  Further, it will be revisited in the succeeding
shower reconstruction algorithms.

% -----------------------------------------------------------------------
\subsection{Geometry}

% -----------------------------------------------------------------------
\subsection{Starting Point}

% -----------------------------------------------------------------------
\subsection{Calorimetry}

